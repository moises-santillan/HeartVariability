% ****** Start of file apssamp.tex ******
%
%   This file is part of the APS files in the REVTeX 4.2 distribution.
%   Version 4.2a of REVTeX, December 2014
%
%   Copyright (c) 2014 The American Physical Society.
%
%   See the REVTeX 4 README file for restrictions and more information.
%
% TeX'ing this file requires that you have AMS-LaTeX 2.0 installed
% as well as the rest of the prerequisites for REVTeX 4.2
%
% See the REVTeX 4 README file
% It also requires running BibTeX. The commands are as follows:
%
%  1)  latex apssamp.tex
%  2)  bibtex apssamp
%  3)  latex apssamp.tex
%  4)  latex apssamp.tex
%
\documentclass[%
% reprint,
%superscriptaddress,
%groupedaddress,
%unsortedaddress,
%runinaddress,
%frontmatterverbose, 
preprint,
%preprintnumbers,
%nofootinbib,
%nobibnotes,
%bibnotes,
 amsmath,amssymb,
 aps,
%pra,
%prb,
%rmp,
%prstab,
%prstper,
%floatfix,
]{revtex4-2}

\usepackage{graphicx}% Include figure files
\usepackage{dcolumn}% Align table columns on decimal point
\usepackage{bm}% bold math
%\usepackage{hyperref}% add hypertext capabilities
%\usepackage[mathlines]{lineno}% Enable numbering of text and display math
%\linenumbers\relax % Commence numbering lines

%\usepackage[showframe,%Uncomment any one of the following lines to test 
%%scale=0.7, marginratio={1:1, 2:3}, ignoreall,% default settings
%%text={7in,10in},centering,
%%margin=1.5in,
%%total={6.5in,8.75in}, top=1.2in, left=0.9in, includefoot,
%%height=10in,a5paper,hmargin={3cm,0.8in},
%]{geometry}

\begin{document}

\preprint{}

\title{Heart stroke-volume variability in a murine model for heart failure with reduced ejection fraction}% Force line breaks with \\
%\thanks{A footnote to the article title}%

\author{Gemma Fernández-Mendoza}
 \email{gemmag.fernandez@gmail.com}
\affiliation{%
 Departamento de Física, Escuela Superior de Física y Matemáticas, Instituto Politécnico Nacional, 07738 Ciudad de México, México
}%

\author{Moisés Santillán}
 \homepage{http://moises-santillan.github.io}
 \email{msantillan@cinvestav.mx}
\affiliation{
 Centro de Investigación y de Estudios Avanzados, Unidad Monterrey, 66628 Apodaca NL, México
}%

\date{\today}% It is always \today, today,
             %  but any date may be explicitly specified

\begin{abstract}
Write an abstract
\end{abstract}

%\keywords{Suggested keywords}%Use showkeys class option if keyword
                              %display desired
\maketitle

%\tableofcontents

\section{\label{sec:intro}Introduction}

\section{Materials and Methods}

\section{Results}

\subsection{Experimental Results}

\subsection{Variability analysis}

Heart rate variability has been measured using a variety of techniques. The majority of them are based on the notion of signal stationarity. However, the heart rate's inherently non-stationary nature---which undergoes continuous physiological change to adapt to outside stimuli---presents a significant challenge that could lead to inaccurate results \citep{Marwan_2007}. Although a number of signal preprocessing methodologies have been suggested to address these problems, nonlinear analysis-based strategies are frequently used and seem to produce reliable result \citep{Marwan_2002, Aubert_2003, Marwan_2007, Giuliani_1998, Rajendra_Acharya_2006, Webber_1994, Henriques_2020}. One of them that is used in different scientific domains is the Poincairé plot \citep{Hoshi_2016, Webber_1994}, which has numerous applications in different scientific domains \citep{Voss_2008}. In a Poincairé plot, all values of a time series are plotted against previous values, leading to an ellipsoidal point cloud. Moreover, this diagram may be analyzed quantitatively by the standard deviations of the point projections along the lines $y = x$ (SD2) and $y = -x$ (SD1). The transverse axis of the ellipse (SD1) is a measure of the short-term changes in the the time series, while the longitudinal axis (SD2) reflects long-term changes. In the particular case of heart-beat duration time-series, SD1 is considered as an indicator of the parasympathetic activity, whereas SD2 is considered as an inverse indicator of the sympathetic activity \citep{Zimatore_2022}.

\subsection{Mathematical Model}

\begin{figure}
\includegraphics[width=4in]{model.pdf}
\caption{Schematic representation of the model}
\label{fig:model}
\end{figure}

\section{Concluding Remarks}

\begin{acknowledgments}
We wish to acknowledge the support of the author community in using
REV\TeX{}, offering suggestions and encouragement, testing new versions,
\dots.
\end{acknowledgments}



\bibliography{manuscript}% Produces the bibliography via BibTeX.

\end{document}
%
% ****** End of file apssamp.tex ******
